\documentclass[conference]{IEEEtran}
\IEEEoverridecommandlockouts
% The preceding line is only needed to identify funding in the first footnote. If that is unneeded, please comment it out.
\usepackage{cite}
\usepackage{amsmath,amssymb,amsfonts}
% \usepackage{amsmath}
\usepackage{graphicx}
\usepackage{textcomp}
\usepackage{xcolor}
\usepackage{url} 
\usepackage{subcaption} % Required for creating subfigures


%% CAPTION FORMAT %%%%%%%%%%%%%%%%%%%%%%%%
\usepackage{tabularx}
\usepackage{xcolor}
\usepackage{algorithm}
\usepackage{algpseudocode}
\usepackage{caption}
\usepackage{array} % for better columns control
\usepackage{multirow} % for multirow feature
\captionsetup[figure]{labelfont=bf}
\renewcommand{\figurename}{Figure}
\captionsetup[table]{labelfont=bf}
\renewcommand{\tablename}{Table}

% bibliography style
\bibliographystyle{IEEEtranN}
\usepackage[numbers,sort&compress]{natbib}

\begin{document}
% \setlength{\parskip}{1pt plus 0pt minus 1pt}


\title{PV optimal location based on power conversion efficiency and irradiance distribution in the residential sector\\
% \thanks{Identify applicable funding agency here. If none, delete this.}
}

\author{\IEEEauthorblockN{Alexis Aguilar Celis}
\IEEEauthorblockA{\textit{Engineering Department} \\
\textit{Durham University}\\
Durham, United Kingdom \\
alexis.a.aguilar-celis@durham.ac.uk}
\and
\IEEEauthorblockN{Hongjian Sun}
\IEEEauthorblockA{\textit{Engineering Department} \\
\textit{Durham University}\\
Durham, United Kingdom \\
hongjian.sun@durham.ac.uk}
\and
\IEEEauthorblockN{Christopher Groves}
\IEEEauthorblockA{\textit{Engineering Department} \\
\textit{Durham University}\\
Durham, United Kingdom \\
chris.groves@durham.ac.uk}
\and
\IEEEauthorblockN{Pratik Harsh}
\IEEEauthorblockA{\textit{Engineering Department} \\
\textit{Durham University}\\
Durham, United Kingdom \\
pratik.harsh@durham.ac.uk}
}
\maketitle
\begin{abstract}
  % Increased photovoltaic (PV) penetration in the low-voltage residential sector highlights the intrinsic problems with Silicon PV (Si-PV) in seasonal power output fluctuations.
  % Exploring the potential performance of alternative PV materials, such as Perovskite and Organic in low-light conditions could significantly enhance residential energy systems. 
  % In this research, a novel material model is presented to evaluate the performance of Organic PV materials across various temperatures and light conditions. 
  % The findings indicate that Low-Light Enhanced PV (LLE-PV) reduces grid energy dependency by 66.37\% in autumn and 22.85\% in winter.
  % Consequently, this boosts the self-sufficiency ratios to 0.83 in autumn and 0.39 in winter, in contrast to Si-PV's ratios of 0.59 and 0.21, respectively.
  % Additionally, it enabled the transfer of excess energy to the grid during autumn, period significantly affected by solar irradiance variability, an achievement unattainable with Silicon PV.
\end{abstract}

\begin{IEEEkeywords}
  BESS, Energy self-sufficiency, Low-Light Solar Cells, PV, Renewable energy integration.
\end{IEEEkeywords}


\input{content/background.tex}
\section*{Solar Irradiance Calculation on a Tilted Surface Using the Hay-Davies Method}

The calculation of solar irradiance on a tilted surface using the Hay-Davies method involves several steps, starting with the determination of solar position, estimating clear sky irradiances, and finally applying the Hay-Davies method to compute the total irradiance. Here are the detailed steps and equations used in the process:

\subsection*{Solar Position Calculation}
The solar zenith and azimuth angles are critical for determining the amount of solar radiation that reaches the Earth's surface at a specific location and time. They are calculated using the following equations:

\begin{align}
\text{Solar Zenith Angle, } \theta_z &= \cos^{-1}(\sin \phi \sin \delta + \cos \phi \cos \delta \cos h) \\
\text{Solar Azimuth Angle, } \Psi &= \cos^{-1}\left(\frac{\sin \delta - \sin \phi \cos \theta_z}{\cos \phi \sin \theta_z}\right)
\end{align}

where $\phi$ is the latitude, $\delta$ is the solar declination, and $h$ is the hour angle.

\subsection*{Clear Sky Irradiance Estimation}
Using the solar position, we estimate the clear sky global horizontal irradiance ($I_{\text{ghi}}$) and direct normal irradiance ($I_{\text{dni}}$), often using empirical models such as the Ineichen model:

\begin{align}
I_{\text{dni}} &= I_{\text{ext}} e^{-k / \cos \theta_z} \\
I_{\text{ghi}} &= I_{\text{dni}} \cos \theta_z + I_{\text{dhi}}
\end{align}

where $I_{\text{ext}}$ is the extraterrestrial irradiance, $k$ is an atmospheric clarity coefficient, and $I_{\text{dhi}}$ is the diffuse horizontal irradiance.

\subsection*{Hay-Davies Method for Tilted Surfaces}
The Hay-Davies method calculates the total irradiance on a tilted surface as follows:

\begin{align}
I_{t, \text{total}} &= I_{d, \text{isotropic}} + I_{d, \text{circumsolar}} + I_{b, \text{direct}} \cos \theta_i \\
I_{d, \text{isotropic}} &= I_{h, \text{diffuse}} \frac{1 + \cos \beta}{2} \\
I_{d, \text{circumsolar}} &= I_{b, \text{direct}} \times \text{Circumsolar Factor} \\
I_{b, \text{direct}} &= I_{\text{dni}} \cos \theta_i
\end{align}

where:
\begin{itemize}
  \item $\theta_i$ is the angle of incidence of the solar beam on the tilted surface,
  \item $\beta$ is the tilt angle of the surface,
  \item $\text{Circumsolar Factor}$ is a model-dependent constant, often taken as a small fraction of $I_{b, \text{direct}}$.
\end{itemize}

The cosine of the angle of incidence, $\cos \theta_i$, is calculated using:
\begin{equation}
\cos \theta_i = \sin \theta_z \cos \beta \cos(\Psi - \gamma) + \cos \theta_z \sin \beta
\end{equation}

where $\gamma$ is the azimuth angle of the surface.

\subsection*{Conclusion}
The Hay-Davies method provides a detailed and accurate approach for calculating solar irradiance on tilted surfaces, incorporating effects of both the isotropic and anisotropic components of sky irradiance. This method is particularly useful in photovoltaic simulation and performance modeling.

\bibliography{Content/references}


\end{document}
