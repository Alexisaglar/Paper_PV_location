\section{Irradiance Estimation Models}

The simulation of solar irradiance utilizes a combination of models provided by PVLIB, which are based on established photovoltaic research. The main models and their respective equations are described below.

\subsection{Clear Sky Model (Ineichen)}
The Ineichen clear sky model, used for estimating clear sky Global Horizontal Irradiance (GHI), Direct Normal Irradiance (DNI), and Diffuse Horizontal Irradiance (DHI), is represented by the following empirical formula:

\begin{equation}
I = I_0 \cdot e^{-k \cdot m}
\label{eq:ineichen}
\end{equation}
where \( I_0 \) is the extraterrestrial irradiance, \( k \) is an attenuation coefficient that varies with atmospheric conditions, and \( m \) is the relative air mass.

\subsection{Erbs Model}
The Erbs model is employed to decompose GHI into DNI and DHI based on the solar zenith angle. The model is particularly useful under conditions where only GHI measurements are available. It is given by:

\begin{align}
DNI &= \frac{GHI - DHI}{\cos(\text{zenith})} \\
DHI &= GHI \cdot d(\text{zenith})
\label{eq:erbs}
\end{align}
where \( d(\text{zenith}) \) is a function defining the fraction of diffuse irradiance as a function of the solar zenith angle. This function is derived from empirical data and provides a way to estimate DHI from GHI.

\subsection{Plane of Array Irradiance Calculation}
The total irradiance on a plane oriented at a specific tilt and azimuth is calculated by considering the contributions from direct, diffuse, and ground-reflected components. The relevant equations are:

\begin{align}
POA_{direct} &= DNI \cdot \cos(\theta) \\
POA_{diffuse} &= DHI \cdot \frac{1 + \cos(\phi)}{2} \\
POA_{reflected} &= GHI \cdot \rho \cdot \frac{1 - \cos(\phi)}{2}
\label{eq:poa_irradiance}
\end{align}
where \( \theta \) is the angle of incidence of the solar radiation, \( \phi \) is the tilt angle of the panel, and \( \rho \) is the albedo or ground reflectance.

These equations collectively allow for the calculation of the total Plane of Array (POA) irradiance, integrating contributions from all solar components, adjusted for specific panel configurations.

\section{Conclusion}
By applying these models, the simulation accurately represents the irradiance received by photovoltaic panels under various tilt and azimuth conditions, thus enabling optimized design and placement of solar installations.
\section{Utilization of POA Irradiance for PV Generation Calculation}

\subsection{Components of POA Irradiance}
The total Plane of Array (POA) irradiance, utilized for calculating power output from a photovoltaic panel, is composed of three distinct parts: direct, diffuse, and reflected irradiance. These components are calculated as follows:

\begin{align}
POA_{\text{direct}} &= DNI \cdot \cos(\theta) \\
POA_{\text{diffuse}} &= DHI \cdot \frac{1 + \cos(\phi)}{2} \\
POA_{\text{reflected}} &= GHI \cdot \rho \cdot \frac{1 - \cos(\phi)}{2}
\end{align}

where:
\begin{itemize}
    \item \( \theta \) is the angle of incidence of the solar radiation on the panel,
    \item \( \phi \) is the tilt angle of the panel,
    \item \( \rho \) is the ground reflectance (albedo).
\end{itemize}

\subsection{Power Output Estimation}
The total POA irradiance, which includes all three components (direct, diffuse, and reflected), is used to estimate the power output of a photovoltaic system:

\begin{equation}
P_{\text{out}} = \eta \cdot A \cdot (POA_{\text{direct}} + POA_{\text{diffuse}} + POA_{\text{reflected}}) \cdot PR
\label{eq:pv_power_output}
\end{equation}

This equation highlights the influence of all aspects of solar irradiance on the panel. Specifically, it takes into account how the panel's orientation (tilt and azimuth) and environmental factors like ground reflectance affect the total energy available for conversion into electricity. The formula integrates the contributions as follows:

\begin{itemize}
    \item \( POA_{\text{direct}} \) captures the irradiance from the solar disc directly hitting the panel, maximized when the panel directly faces the sun.
    \item \( POA_{\text{diffuse}} \) accounts for the sky-diffused solar radiation, which is less dependent on the panel's exact angle to the sun but varies with sky clarity and sun position.
    \item \( POA_{\text{reflected}} \) includes the irradiance reflected from the ground or nearby surfaces onto the panel, influenced by the ground material and the panel's tilt.
\end{itemize}

\subsection{Efficiency and System Losses}
The conversion efficiency (\( \eta \)), area of the PV module (\( A \)), and performance ratio (\( PR \)) remain critical to determining the overall system output, incorporating efficiency losses due to factors such as temperature, inverter efficiency, and installation quality.

\subsection{Conclusion}
By accounting for all components of POA irradiance in the power output calculations, the simulation provides a comprehensive assessment of potential energy production, allowing for optimized system design and better prediction of energy yield under various environmental and installa
